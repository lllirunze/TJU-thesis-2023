\cabstract{
中文摘要一般在~400~字以内,简要介绍毕业论文的研究目的、方法、结果和结论,语言力求精炼。中英文摘要均要有关键词,一般为~3~—~7~个。字体为小四号宋体,各关键词之间要有分号。英文摘要应与中文摘要相对应,字体为小四号~Times New Roman,详见模板。

此处格式已按模板设定,作者只需选择段落区域,输入替换之。模板中所有说明性文字用于注释格式与内容的要求,撰写论文时请删除。

中文摘要一般为~300~-~400~字,简要介绍毕业设计(论文)的研究目的、方法、结果和结论,语言力求精炼。英文摘要应与中文摘要相对应。中英文摘要均要有关键词,一般为~3~-~8个,中英文摘要要相互对应。

中文摘要。“摘要”两字之间空一个全角空格或两个半角空格,字体为宋体二号字加粗,居中显示,摘要内容采用正文样式。中文关键词与摘要内容间隔一行,无缩进左对齐书写。“关键词:”采用宋体四号字加粗,关键词内容采用正文样式,且换行不缩进。关键词之间用逗号分隔。

英文摘要。此部分皆为~Times New Roman~字体。“~ABSTRACT~”为二号字加粗,居中显示。英文摘要内容采用正文样式。英文关键词与英文摘要内容间隔一行,无缩进左对齐书写。“~KEY WORDS:~”为四号字加粗,英文关键词采用正文样式,且换行不缩进,关键词之间用逗号分隔,词义和中文关键词相同。“~ABSTRACT~”和“~KEY WORDS~”一律用大写字母,每个关键词的首字母要大写。

}

\ckeywords{关键词~1,关键词~2,关键词~3,……,关键词~7(关键词总共~3~—~7~个,最后一个关键词后面没有标点符号)}

\eabstract{
The upper bound of the number of Chinese characters is 400. The abstract aims at introducing the research purpose, research methods, research results, and research conclusion of graduation thesis, with refining words. Generally speaking, both the Chinese and English abstracts require the keywords, the number of which varies from 3 to 7, with a semicolon between adjacent words. The font of the English Abstract is Times New Roman, with the size of 12pt(small four).
}

\ekeywords{keyword 1, keyword 2, keyword 3, ……, keyword 7 (no punctuation at the end)}

\makecover

\clearpage